\documentclass[10pt,usenames,dvipsnames]{beamer}
\usepackage{swjtu}
\usepackage[]{biblatex}
%\setbeamertemplate{bibliography item}{\insertbiblabel}
\addbibresource{bibliography.bib}
\usepackage[spanish]{babel}
\usepackage{amsmath} 
\usepackage{amsfonts}
\usepackage{amssymb}
\usepackage{amsthm}
\usepackage{xcolor}
%\usepackage{mdframed}
\usepackage{inputenc}
\usefonttheme{professionalfonts}
%\usefonttheme{serif}
\usepackage{graphicx}
\setbeamercovered{transparent}
\usepackage{ragged2e}
\usepackage{circuitikz}
\useinnertheme{tcolorbox}
\justifying{

\title{\textbf{1. Repaso de Estadística}}
\subtitle{Econometría Avanzada}
\author[Daniel Kelly (Colmex)]{Daniel Kelly}
\institute[Colmex]{El Colegio de México}
\date{Invierno 2023}

\begin{document}
%% Title page
%% Remove '[plain]' if you need a footline
\begin{frame}[plain]
    \titlepage
\end{frame}

{
\begin{frame}[plain]
    \frametitle{Contenido}
    \tableofcontents
\end{frame}
}

\begin{frame}[plain]
\textbf{Lectura:} Wooldridge (2001), \textit{Econometric Analysis of Cross-Sectional and Panel Data}, Capítulos 1-3.
\end{frame}

\section{Introducción}

\subsection{Conceptos Preeliminares}

\begin{frame}{Conceptos Preeliminares}
Sea $y$ una variable aleatoria continua con P.D.F. (función de densidad) dada por $f(y,\theta)$:

\bigskip
\begin{exampleblock}{Muestra Independiente e Idénticamente Distribuida}
Sí ${Y_1,\hdots,Y_n}$ es una sucesión de variables aleatorias independientes con P.D.F. común $f(Y_i)$; entonces ${Y_1,\hdots,Y_n}$ es una muestra \textbf{Independiente y Idénticamente Distribuida}.
\end{exampleblock}
\end{frame}

\begin{frame}
    La esperanza condicional de $y$ dado $\x_0$ (un valor particular de $\x$) está dada por 
\eq{
    \E{y|\x=\x_0} = \dfrac{1}{f_\x (\x_0)}\int_{-\infty}^{\infty} y f_{\x,y}(y,\x_0)
}
\pause

\bigskip
Si la esperanza de $y$ es finita ($\E{y|x_1,\hdots,x_k}<\infty$), podemos decir que toma una forma funcional específica:
\eq{
    \E{y|x_1,\hdots,x_k}=\mu(x_1,\hdots,x_k) \text{ con } \mu:\R^k \to \R
}
\end{frame}

\begin{frame}{Ley de Esperanzas Iteradas}
    Un concepto particularmente útil en econometría es la Ley de Esperanzas Iteradas:

\begin{exampleblock}{Ley de Esperanzas Iteradas}
    Sea $\mn{w}$ un vector aleatorio. A partir de este vector, definimos una función $\x=f(\mn{w})$. La \textbf{Ley de Esperanzas Iteradas} entonces dice que:
    \eq{\E{y|\x}=\E{\E{y|\mn{w}}|\x}}
    Una forma alternativa de frasear la \textbf{LEI} tiene que ver con las medias. Definimos $\mu_1(\mn{w})\equiv \E{y|\mn{w}}$ y  $\mu_2(\x)\equiv \E{y|\mn{x}}$. Entonces:
    \eq{\mu_2(\x)=\E{\mu_1(\mn{w})|\x}}
\end{exampleblock}
\end{frame}

\begin{frame}
De acuerdo con la teoría que respalde la estimación, $\mu(\cdot)$ puede ser lineal o no, y refleja el cambio en el valor esperado de $y$ ante un cambio en $x_j$, es decir, el \textbf{efecto parcial}:
\eq{
    \D \E{y|\x} \approx \pd{ \mu(\x) }{ x_j } \cdot \D x_j
}    

\pause
\bigskip
La derivada de $\mu(\x)$ con respecto a $x_j$ se puede interpretar con el efecto parcial de un cambio unitario en $x_j$ sobre la esperanza condicional ($\D x_j=1$).
\end{frame}

\begin{frame}{Elasticidad}
    En el caso de la \textit{elasticidad}, los cambios se miden en términos porcentuales. Recordemos la deifnición:
    \eq{
    e_{y,x_j} = \pd{ \mu(\x) }{ x_j } \cdot \dfrac{ x_j }{ \mu(\x) }
}
\end{frame}

%% Final page
\begin{frame}[plain]
    \begin{picture}(0,0)
        \put(-21.5, -164){\includegraphics[width=\paperwidth]{Imágenes/final_page_bg.png}}
    \end{picture}
\end{frame}

\end{document}
