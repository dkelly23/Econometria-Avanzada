\chapter{Repaso de Estadística}

\section{Introducción}

Sea $y$ una variable aleatoria continua con P.D.F. (función de densidad) dada por $f(y,\theta)$:

\bigskip
\defi{Muestra Independiente e Idénticamente Distribuida}{
    Sí ${Y_1,\hdots,Y_n}$ es una sucesión de variables aleatorias independientes con P.D.F. común $f(Y_i)$; entonces ${Y_1,\hdots,Y_n}$ es una muestra \textbf{Independiente y Idénticamente Distribuida}. 
}

\bigskip
Dado un vector de variables independientes $\x=(x_1,\hdots,\x_k)$, podemos definir la esperanza condicional de $y$ como: $\E{y|\x}$. La esperanza condicional de $y$ dado $\x_0$ (un valor particular de $\x$) está dada por 
\eq{
    \E{y|\x=\x_0} = \dfrac{1}{f_\x (\x_0)}\int_{-\infty}^{\infty} y f_{\x,y}(y,\x_0)
}

\bigskip
Si la esperanza de $y$ es finita ($\E{y|x_1,\hdots,x_k}<\infty$), podemos decir que toma una forma funcional específica:
\eq{
    \E{y|x_1,\hdots,x_k}=\mu(x_1,\hdots,x_k) \text{ con } \mu:\R^k \to \R
}

\bigskip
\defi{Ley de Esperanzas Iteradas}{
    Sea $\mn{w}$ un vector aleatorio. A partir de este vector, definimos una función $\x=f(\mn{w})$ ($\x$ incluso podría ser solo un subconjunto de $\mn{w}$). Esto implica que al conocer los valores de $\mn{w}$, conocemos todos los valores de $\x$. La \textbf{Ley de Esperanzas Iteradas} entonces dice que:
    \eq{\E{y|\x}=\E{\E{y|\mn{w}}|\x}}
    Una forma alternativa de frasear la \textbf{LEI} tiene que ver con las medias. Definimos $\mu_1(\mn{w})\equiv \E{y|\mn{w}}$ y  $\mu_2(\x)\equiv \E{y|\mn{x}}$. La \textbf{LEI} implica que podemos obtener el valor de $\mu_2(\x)$ obteniendo el valor esperado de $\mu_1(\w)$ dado el vector $\mn{w}$:
    \eq{\mu_2(\x)=\E{\mu_1(\mn{w})|\x}}
    Igualmente, la \textbf{LEI} tiene otra implicación parecida a (1.3):
    \eq{\E{y|\x}=\E{\E{y|\x}|\mn{w}}}
}

\bigskip
De acuerdo con la teoría que respalde la estimación, $\mu(\cdot)$ puede ser lineal o no, y refleja el cambio en el valor esperado de $y$ ante un cambio en $x_j$, es decir, el efecto parcial:
\eq{
    \D \E{y|\x} \approx \pd{ \mu(\x) }{ x_j } \cdot \D x_j
}

\bigskip
La derivada de $\mu(\x)$ con respecto a $x_j$ se puede interpretar con el efecto parcial de un cambio unitario en $x_j$ sobre la esperanza condicional ($\D x_j=1$). \\

Un caso particularmente relevante en la teoría económica es el caso de la elasticidad, en dónde los cambios se miden en términos porcentuales (i.e. el efecto de un cambio porcentual en $x_j$ sobre un cambio porcentual en $y$). La elasticidad se define como:
\eq{
    e_{y,x_j} = \pd{ \mu(\x) }{ x_j } \cdot \dfrac{ x_j }{ \mu(\x) }
}

\bigskip
Podemos reescribir (1.7) como: \\

$\begin{array}{llllll}
    & &  e_{y,x_j} = \pd{ \mu(\x) }{ x_j } \cdot \dfrac{ x_j }{ \mu(\x) } = 
    \dfrac{ \dfrac{ 1 }{ \mu(\x) } \cdot \pd{ \mu(\x) }{ x_j } }{ \dfrac{ 1 }{ x_j } } = \dfrac{ \pd{ \log \mu(\x) }{ x_j } }{ \pd{ \log x_j }{ x_j } } = \pd{ \log \mu(\x) }{ \log \x_j }
\end{array}$

\bigskip
Esto es, podemos reescribir la elasticidad como el cociente de las derivadas de los logaritmos naturales de $\mu(\x)$ y $x_j$:
\eq{
    e_{y,x_j}=\pd{\log \mu(\x)}{\log x_j}
}


\section{Especificación del Error}

Si $y$ es una Variable Aleatoria, podemos escribirla como:
\eq{
    y=\E{y|\x}+\varepsilon
}

\bigskip
Dónde $\varepsilon$ es un término de error que captura todos los factores no observables que afectan a $y$. En ese sentido, los supuestos de identificación son en realidad supuestos sobre $\varepsilon$. Normalmente suponemos:
\eq{
    \E{\varepsilon | \x} = 0 \hspace{1cm} \text{\blue{(Media Condicional Cero)}}
}

\bigskip
\textbf{Implicaciones:}

\begin{enumerate}
    \item $\E{\varepsilon}=0$
\end{enumerate}

\proof{
Por la Ley de Esperanzas Iteradas:
$$\E{\varepsilon}=\E{\E{\varepsilon|\x}}$$
En tanto la esperanza condicional de $\varepsilon$ es cero, la esperanza no condicionada también lo es. Esto es una aplicación de la \textit{Ecuación 1.3}.
}

\bigskip
\begin{enumerate}[resume]
    \item $\cov{\varepsilon, x_j}=0, \hspace{1cm} \forall x_j \in \x$
\end{enumerate}

\proof{
Desarrollamos la covarianza para obtener: \\

\begin{array}{llllllll}
     &  & \cov{\varepsilon, x_j} & = & \E{\varepsilon\cdot x_j} - \E{\varepsilon} \E{x_j} \\
     \\
     &  & & = & \E{\varepsilon\cdot x_j}
\end{array}

\bigskip
Igualmente haciendo uso de la \textit{Ecuación 1.3} obtenemos: \\

\begin{array}{llllllll}
     &  & \E{\varepsilon\cdot x_j} & = & \E{\E{\varepsilon\cdot x_j|\x}} \\
     \\
     &  & & = & \E{x_j\E{\varepsilon|\x}} = 0 
\end{array}

\bigskip
En consecuencia, la covarianza entre $x_j$ (un elemento genérico de $\x$) y $\varepsilon$ es siempre igual a cero.
}

\bigskip
\begin{enumerate}[resume]
    \item $\cov{\varepsilon, f(x_j)}=0, \hspace{0.5cm} \forall x_j \in \x$ \hspace{1cm} (El argumento es estrictamente análogo al anterior).
\end{enumerate}

\bigskip
Normalmente, asumir que el error posee Media Condicional Cero (\textit{Ecuación 1.10}) es estrictamente equivalente a decir que el modelo está bien identificado. Esto quiere decir que, dados los valores de las variables independientes capturadas en $\x$, el error no posee ninguna información adicional que capture relaciones entre las variables del modelo. \\

Sin embargo, esta es una suposición muy fuerte, por lo que normalmente suponemos que únicamente (2.) y (3.) se cumplen. Hago notar que esto no implica que el supuesto de Media Condicional Cero se cumpla. 

\section{Proyecciones Lineales}

Hemos establecido ya que el problema de interés es estimar la Esperanza Condicional (\textit{Ecuación 1.2}), sin embargo, la forma funcional desconocida impone problemas en la estimación (tanto estadísticos como computacionales). Para transformar esta Esperanza Condicional en un modelo que podamos estimar, utilizamos la Proyección Lineal de $y$ sobre $\x$.\\

\defi{Proyección Lineal}{
La proyección lineal de $y$ sobre el espacio generado por el vector $\x=(1,x_2,\hdots,x_K)$ está dada por una combinación lineal de \x:
\eq{
    L(y|\x) = \dsum{i=1}{K}\delta_i x_i = \delta_1 + \delta_2 x_2 + \hdots + \delta_K x_K
    }
Dónde:
\eq{
    \d=\var{\x}^{-1}\cdot\cov{\x,y}
    }
    
\bigskip
En la \textit{Ecuación 1.12}, $\var{\x}$ es la matriz de varianza-covarianza de $\x$ (dimensión $K\times K$) y $\cov{\x,y}$ es el vector de covarianzas de $\x$ y $y$ (dimensión $K\times K$). Nótese que esta estimación es análoga al estimador por M.C.O. del vector de parámetros.
}

\bigskip
En el estimador del vector de coeficientes de la proyección lineal:
\begin{itemize}
    \item El elemento $\lbrace i,j \rbrace$ de $\var{\x}$ es igual a $\cov{x_i,x_j}$.
    \item El $i$-ésimo elemento de $\cov{\x,y}$ es $\cov{x_i,y}$.
\end{itemize}

\bigskip
Supongamos que $\lbrace y,x_1,\hdot,x_K \rbrace$ es un conjunto de Variables Aleatorias con esperanza finita tal que $\E{y^2}<\infty$ y $\E{x_{j}^2}<\infty$, $\forall x_j \in \x$. En particular:
\begin{itemize}
    \item No hay restricciones conjuntas sobre la distribución de $\x$ y $y$.
    \item $\var{\x}$ es una matriz de $K\times K$ no singular y positiva definida.
\end{itemize}

\bigskip
Entonces, la Proyección Lineal de $y$ sobre $\x$ siempre existe y es única. Dado $L(y|\x)$, podemos reescribir $y$ como:
\eq{
    y=L(y|\x)+r
}

\bigskip
La Proyección Lineal cumple propiedades análogas a las del estimador por M.C.O:
\begin{itemize}
    \item $\E{r}=0$
    \item $\cov(r, x_j)=0$
    \item $\E{r^2}<\infty$
\end{itemize}


\section{Teoría Asintótica}

\defi{Convergencia en Probabilidad}{
Una sucesión de Variables Aleatorias $\lbrace x_N:N=1,2,\hdots \rbrace$ converge en probabilidad a la constante $\a$ si para todo $\e>0$:
\eq{
    \text{Prob}(|x_N-\a|>\e) \to 0 \hspace{1cm} \text{cuando} \hspace{0.2cm} N\to\infty
    }
}

\bigskip
A la idea de Convergencia en Probabilidad se le llama también \textit{Probability Limit}:
\eq{
x_N \ov{\to}{P} \a \hspace{0.5cm} \text{o} \hspace{0.5cm} \plim{x_N}=\a 
}

\bigskip
Es decir, cuando $N$ tiende a infinito, la probabilidad de que la sucesión esté arbitrariamente cerca de $\a$ tiende a 1.\\

\teo{Teorema de Slutsky}{
Sea $g:\R^K\to\R^J$ una función continua alrededor de $c\in\R^J$. Si $\x_N$ es una sucesión de Vectores Aleatorios de $K\times 1$ tal que $\plim{\x_N}=c$. Entonces:
\eq{g(\x_N) \ov{\to}{P} g(c) \hspace{1cm} \text{cuando} \hspace{0.2cm} N\to\infty}
\eq{\plim{g(\x_N)}=g(c)=g(\plim {\x_N})}
}

\bigskip
El Teorema de Slutsky dice que bajo las condiciones de continuidad y convergencia establecidos antes, la convergencia en probabilidad funciona como un operador lineal. En particular, si $\a_N$ y $\b_N$ son estimadores de los parámetros $\a$ y $\b$ consistentes asintóticamente, es decir:
\eq{
    \plim{\a_N}=\a \hspace{0.5cm} \text{y} \hspace{0.5cm} \plim{\b_N}=\b
}

\bigskip
En consecuencia:
\begin{enumerate}
    \item $\plim{\a_N+\b_N}=\a+\b$
    \item $\plim{\a_N\cdot\b_N}=\a\cdot\b$
    \item $\plim{\dfrac{\a_N}{\b_N}}=\dfrac{\a}{\b}$
\end{enumerate}

\bigskip
\defi{Convergencia en Distribución}{
Una sucesión de Variables Aleatorias $\lbrace x_N:N=1,2,\hdots \rbrace$ converge en distribución a la variable continua $x$ sí y solo sí:
\eq{
    F_N(\xi) \to F(\xi) \hspace{1cm} \text{cuando} \hspace{0.2cm} N\to\infty, \hspace{0.2cm} \forall \xi \in \R
    }   
Dónde $F_i(\cdot)$ es la C.D.F. de $x_n$ y $F(\cdot)$ es la C.D.F. de $x$.
}

\bigskip
La convergencia en distribución se denota $x_N\ov{\to}{d}x$. Cuando $N\to\infty$, la distribución de $x_N$ tiende a la de $x$ (decimos también que se distribuye asintóticamente como $x$). Por ejemplo, cuando $x_N$ se distribuye asintóticamente normal la denotamos como:
\eq{
    x_N \ov{\to}{d} N(\mu,\sigma^2) \hspace{0.5cm} \text{o} \hspace{0.5cm} x_N \ov{\to}{a}N(\mu,\sigma^2)
}

\bigskip
Supongamos que $\x_N$ es una sucesión de Vectores Aleatorios de $K\times 1$ tal que $\x_N\ov{\to}{d}\x$. Si $g:\R^K\to\R^J$ es una función continua, entonces:
\eq{
    g(\x_N) \ov{\to}{d} g(\x)
}

Es decir, la convergencia en distribución se mantiene entre funciones continuas. Los siguientes resultados giran en torno a la distribución asintótica de los estimadores de los momentos de una distribución.\\

\teo{Ley de los Grandes Números}{
Sea $\lbrace \w_i:i=1,2,\hdots \rbrace$ una sucesión de Vectores Aleatorios de $G\times 1$ i.i.d. con esperanza finita:
$$\E{|w_{ig}|<\infty} \hspace{1cm} \text{para} \hspace{0.25cm} g=1, \hdots,  G$$

\bigskip
Entonces, la sucesión satisface la \textbf{Ley de los Grandes Números} (L.L.N.), es decir:
\eq{\dfrac{1}{N}\dsum{i=1}{N}\w_i\ov{\to}{d}\mu_\w \hspace{0.5cm} \text{ dónde } \hspace{0.5cm} \mu_\w=E[\w_i]}
}
\bigskip

\begin{teorema}{Teorema de Lindberg-Levy}
Sea $\lbrace \w_i:i=1,2,\hdots \rbrace$ una sucesión de Vectores Aleatorios de $G\times 1$ i.i.d. con esperanza finita:
\eq{\E{|w_{ig}|}<\infty \hspace{0.5cm} \text{ y } \hspace{0.5cm} E[\w_i]=0}

\bigskip
Entonces, la sucesión sigue el \textbf{Teorema del Limite Central} (C.L.T.), es decir:
\eq{\dfrac{1}{\sqrt{N}}\dsum{i=1}{N}\w_i \dist \nor{0}{\var{w_i}}}
\end{teorema}

\section{Propiedades Asintóticas de los Estimadores}

\subsection{Consistencia}

Sea $\lbrace \hat{\t}_N:N=1,2,\hdots \rbrace$ una sucesión de estimadores del vector de parámetros $\t$, dónde $N$ es el tamaño muestral. Decimos que $\hat{\t}_N$ es un estimador \textbf{asintóticamente consistente} de $\t$ si para todo valor de $\t$:
\eq{\plim{\hat{\t}_N}=\t \hspace{0.5cm} \text{cuando} \hspace{0.5cm} N\to\infty }

\bigskip
Es decir, conforme crece $N$ ($N\to\infty$), la distribución de $\hat{\t}_N$ está cada vez más centrada alrededor de $\t$. Nótese que la insesgadez del estimador $\hat{\t}_N$ en general no implica consistencia asintótica. Sin embargo, si $\E{\hat{\t}_N}$ y $\var{\hat{\t}_N}\to 0$ cuando $N\to \infty$, entonces $\plim{\hat{\t}_N}=\t$.

\subsection{Normalidad}

Sea $\lbrace \hat{\t}_N:N=1,2,\hdots \rbrace$ una sucesión de estimadores del vector de parámetros $\t$, dónde $N$ corresponde al tamaño de la muestra. Decimos que $\t_N$ es $\sqrt{N}$ asintóticamente normal si:
\eq{
    \sqrt{N}(\hat{\t}_N-\t) \dist \nor{0}{\mn{V}} 
}

\bigskip
Dónde $\mn{V}$ es la matriz de varianza asintótica de $\sqrt{N}(\hat{\t}_N-\t)$. Es decir, $\text{A}\var{\sqrt{N}(\hat{\t}_N-\t)}=\mn{V}$. La normalización por $\sqrt{N}$ proviene del Teorema del Límite Central. A continuación explicamos porque esta propiedad de la distribución se cumple a partir de dicho teorema. \\

\proof{
El argumento en este caso es análogo al caso de la media que desarrollamos a continuación. Sea $\lbrace \x_N:N=1,2,\hdots \rbrace$ una sucesión de Variables Aleatorias con media $\mu$ y varianza $\sigma^2$. Por el Teorema del Límite Central sabemos que:\\

$\begin{array}{lllllllll}
     &  &  \x_N=\dfrac{1}{N} \dsum{i=1}{N} x_i \hspace{0.25cm} \ov{\to}{d} \hspace{0.25cm} N \( \mu, \dfrac{\sigma^2}{N} \)
\end{array}$

\bigskip
Podemos centrar la distribución restando la media. Posteriormente, normalizamos por la desviación estándar $\sigma/\sqrt{N}$ para obtener:\\

$\begin{array}{lllllllll}
     &  &  \dfrac{\x_N-\mu}{\sigma/\sqrt{N}} =\sqrt{N} \cdot \dfrac{\x_N-\mu}{\sigma}
\end{array}$

\bigskip
El C.L.T. establece que esta cantidad converge en distribución a $N(0, \sigma^2)$. Entonces, eliminando la estandarización tenemos:\\

$\begin{array}{lllllllll}
     &  &  \sqrt{N} (\x_N-\mu) \dist N(0,\sigma^2)
\end{array}$

\bigskip
El mismo resultado se puede generalizar a un estimador genérico \hat{\t} de un parámetro \t. \hspace{0.1cm} 
}

\bigskip
Entonces, tratamos a $\hat{\t}_N$ como si se distribuyera $N(0,\mn{V}/N)$. Esta derivación proviene de la siguiente observación:
\eq{\text{A}\var{\sqrt{N}(\hat{\t}_N-\t)} = N \( A\var{\hat{\t}_N} \) = \mn{V}}

\bigskip
Si tenemos un estimador $\hat{\mn{V}}$ para $\mn{V}$, entonces el estimador:
\eq{\widehat{\text{A}\var{\hat{\t}_N}}=\dfrac{\hat{\mn{V}}}{N}}

\bigskip
\subsection{Eficiencia}
Consideremos dos estimadores $\hat{\t}_N$ y $\Tilde{\t}_N$ del parámetro $\t$ con varianzas asintóticas:\\

$\begin{array}{llllllllllll}
     & & A\var{\sqrt{N}(\hat{\t}_N-\t)} = \mn{V} & & & \& & & & A\var{\sqrt{N}(\Tilde{\t}_N-\t)} = \mn{D}
\end{array}$

\bigskip
Entonces:
\begin{itemize}
    \item $\hat{\t}_N$ es asintóticamente eficiente respecto a $\Tilde{\t}_N$ si $\mn{D}-\mn{V}$ es una matriz positiva semidefinida para cualquier valor de $\t$.
    \item $\hat{\t}_N$ y $\Tilde{\t}_N$ son equivalentes si tienen la misma distribución y la misma varianza asintótica.
\end{itemize}